%%%%%%%%%%%%%%%%%%%%%%%%%%%%%%%%%%%%%%%%%%%%%%%%%%%%%%%%%%%
%\documentclass[xcolor=x11names,compress]{beamer}
\documentclass[xcolor=x11names,compress]{beamer}
%% General document
\usepackage{graphicx, subfig}
%% Beamer Layout
\useoutertheme[subsection=false,shadow]{miniframes}
\useinnertheme{default}
\usefonttheme{serif}
\usepackage{palatino}

%%%%%%% Mes Packages %%%%%%%%%%%%%%%%
%\usepackage[french]{babel}
\usepackage[T1]{fontenc}
\usepackage{color}
\usepackage{xcolor}
\usepackage{dsfont} % Pour indicatrice
\usepackage{url}
\usepackage{multirow}
\usepackage[normalem]{ulem}   % For strike out text

% Natbib for clean bibliography
\usepackage[comma,authoryear]{natbib}

%remove the icon
\setbeamertemplate{bibliography item}{}

%remove line breaks
\setbeamertemplate{bibliography entry title}{}
\setbeamertemplate{bibliography entry location}{}
\setbeamertemplate{bibliography entry note}{}

%% ------ MEs couleurs --------
\definecolor{vert}{rgb}{0.1,0.7,0.2}
\definecolor{brique}{rgb}{0.7,0.16,0.16}
\definecolor{gris}{rgb}{0.7, 0.75, 0.71}
\definecolor{twitterblue}{rgb}{0, 0.42, 0.58}
\definecolor{airforceblue}{rgb}{0.36, 0.54, 0.66}
\definecolor{siap}{RGB}{3,133, 200}


%%%%%%%%%%%%%%%%% BEAMER PACKAGE %%%%%%%

\setbeamercolor{itemize item}{fg=siap}
%\setbeamercolor{itemize subitem}{fg=blue}
%\setbeamercolor{itemize subsubitem}{fg=cyan}

\setbeamerfont{title like}{shape=\scshape}
\setbeamerfont{frametitle}{shape=\scshape}

\setbeamercolor*{lower separation line head}{bg=DeepSkyBlue4}
\setbeamercolor*{normal text}{fg=black,bg=white}
\setbeamercolor*{alerted text}{fg=siap}
\setbeamercolor*{example text}{fg=black}
\setbeamercolor*{structure}{fg=black}
\setbeamercolor*{palette tertiary}{fg=black,bg=black!10}
\setbeamercolor*{palette quaternary}{fg=black,bg=black!10}

% Set the header color to SIAP's color
\setbeamercolor*{frametitle}{fg=siap}

%remove navigation symbols
\setbeamertemplate{navigation symbols}{}

\renewcommand{\(}{\begin{columns}}
\renewcommand{\)}{\end{columns}}
\newcommand{\<}[1]{\begin{column}{#1}}
\renewcommand{\>}{\end{column}}

%% Add footer with logo
\setbeamertemplate{footline}{%
  \begin{beamercolorbox}[wd=\paperwidth,ht=2.5ex,dp=1.125ex,%
    leftskip=.3cm,rightskip=.3cm plus1fil]{author in head/foot}
    \includegraphics[height=5ex]{SIAP_logo_Big.png}\hfill
    \insertshortauthor\hfill\insertshorttitle\hfill  \textcolor{siap}{\textit{\insertframenumber}}
  \end{beamercolorbox}%
}


% Path for the graphs
\graphicspath{{Graphics/}
{../../../../Visualisation/Presentations/Graphics/}
{../../Visualisation/Presentations/Graphics/}
{c:/Gitmain/MLCourse/UNML/Module0/M0_files/figure-html/}
{c:/Chris/UN-ESCAP/MyCourses2022/MLOS2022/Slides/Graphics/}
{c:/Chris/UN-ESCAP/MyCourses2023/BigDataKostat/Slides/Graphics/}
{c:/Chris/UN-ESCAP/MyCourses2024/WebScrapping/Graphics/}
 }

\title{\textcolor{siap}{Machine Learning for Official Statistics in Asia-Pacific countries \\ \vspace{0.5cm} }}

\subtitle{\textcolor{brique}{\Large{Data Science Good Practices}}}
\author{\textcolor{siap}{Christophe Bontemps}}
\institute{ \includegraphics[height=9ex]{SIAP_logo_Big.png}}
\date{}

\begin{document}

\section{Motivation}

\begin{frame}
  \titlepage
\end{frame}


\begin{frame}{Usual practice: Theory vs reality}
\pause
\begin{center}
  \begin{itemize}
        \only<1>{\includegraphics[width=0.8\textwidth]{Process1.png} \\
        You scrapped data from a  website. Saving the file as Excel}
        \only<2>{\includegraphics[width=0.8\textwidth]{Process2.png} \\ Send that file by email to your collaborators.}
        \only<3>{\includegraphics[width=0.8\textwidth]{Process3.png} \\ Someone in your team can start writing some insights.}
        \only<4>{\includegraphics[width=0.8\textwidth]{Process4.png} \\ You, or someone else, start an alaysis ...}
        \only<5>{\includegraphics[width=0.8\textwidth]{Process5.png} \\ ... producing some outputs (graphics, tables, etc..)}
        \only<6>{\includegraphics[width=0.8\textwidth]{Process6.png} \\ But wait... Oh no!  There a bug in the code!}
        \only<7>{\includegraphics[width=0.8\textwidth]{Process7.png} \\ So here is version 2, and another Excel file}
        \only<8>{\includegraphics[width=0.8\textwidth]{Process8.png} \\ And new insights ... }
        \only<9>{\includegraphics[width=0.8\textwidth]{Process9.png} \\ ..and a new analysis based on the second Excell file}
        \only<10>{\includegraphics[width=0.8\textwidth]{Process10.png} \\ And new outputs, new graphics, etc. }
        \only<11>{\includegraphics[width=0.8\textwidth]{Process11.png} \\ Also, using other data sets (classifications, scanner data) }
        \only<12>{\includegraphics[width=0.8\textwidth]{Process12.png} \\ Maybe someone adapts graphics to NSO reports style }
        \only<13>{\includegraphics[width=0.8\textwidth]{Process13.png} \\ Finally,  copy/paste everything into the final report}
        \only<14>{\includegraphics[width=0.8\textwidth]{Process14.png} \\ In the end, this what you have produced!}
    \end{itemize}
\end{center}
\end{frame}

\begin{frame}{Usual practice: In the end}

 \begin{columns}[T]
    \begin{column}{0.5\textwidth}
      \begin{itemize}[<+->]
        \item Lots of files
        \item Cut and paste is not a reliable, reproducible approach!
        \item Your brain may remember (likely not!)...
        \item[] ...all the steps...
        \item[] .. in the right order..
        \item[]...all of them !
        \item Or use (bad) "\emph{tools}"
      \end{itemize}
    \end{column}
    \begin{column}{0.5\textwidth}
       \begin{itemize}
       \item[] \only<1>{\includegraphics[width=0.8\textwidth]{Process14.png} }
        \only<2>{\includegraphics[width=1.0\textwidth]{Process13.png} }
        \only<3>{\includegraphics[width=1.0\textwidth]{Process15.png} }
        \only<4>{\includegraphics[width=1.0\textwidth]{Process16.png} }
        \only<5>{\includegraphics[width=1.0\textwidth]{Process17.png}}
        \only<6>{\includegraphics[width=1.0\textwidth]{Process18.png}}
        \only<7>{\includegraphics[width=1.0\textwidth]{Post-it.jpg} }
    \end{itemize}
    \end{column}
  \end{columns}
\end{frame}

\section{Issues}

\begin{frame}{What are the issues?}
  \begin{columns}[T]
    \begin{column}{0.5\textwidth}
      \begin{itemize}[<+->]
        \item Errors due to cut and paste
        \item Errors are difficult to track
        \item Each operator has his/her own approach
        \item Several versions of code may coexist
        \item The steps aren't recorded
        \item Reproducibility is not granted
       % \item Quality control is hard
      \end{itemize}
    \end{column}
    \begin{column}{0.5\textwidth}
    \begin{itemize}
        \item[]  \only<1-2>{\href{https://www.bbc.com/news/technology-54423988}{ \includegraphics[width=1.0\textwidth]{ExcelUK.png}}}
         \only<3>{\includegraphics[width=1.0\textwidth]{FinalFinalVersion.png}}
         \only<4-5>{\includegraphics[width=1.0\textwidth]{AnatomyOfaFile.png}}
         \only<6>{\includegraphics[width=1.0\textwidth]{Reproducible.jpg}}
         \end{itemize}
    \end{column}
  \end{columns}
\end{frame}

\begin{frame}{Fundamental Principles of Official Statistics}
  \begin{columns}[T]
    \begin{column}{0.6\textwidth}
      \begin{itemize}[<+->]
        \item Clear mention of the \textbf{processes} used to produce statistics
        \item \emph{To retain trust in official statistics, the \textbf{statistical agencies need to decide} according to strictly professional considerations, including scientific principles and professional ethics, \textbf{on the methods and procedures} for the collection, processing, storage and presentation of statistical data.}
      \item In short, \textbf{processes} are important!
      \end{itemize}
    \end{column}
    \begin{column}{0.4\textwidth}
      \includegraphics[width=\textwidth]{FundamentalPrinciplesOS.PNG}
    \end{column}
  \end{columns}
\end{frame}

\section{RAP}  % Better practices

\begin{frame}{Why are Reproducible Analytical Pipelines good for you?}
  \begin{columns}[T]
    \begin{column}{0.7\textwidth}
      \begin{itemize}[<+->]
        \item It will make your (and your team's) (\emph{working}) life easier.
         \item[$\hookrightarrow$]{ \small  Less confusion about where things are and how it works!}
        \item It is an efficient way to work
        \item It helps work faster
        \item It helps make the process of making official statistics more robust!
        \item It makes it easy to efficiently collaborate
        %\item It will enhance your skills (and perhaps make you famous in your organization!)
      \end{itemize}
    \end{column}
    \begin{column}{0.3\textwidth}
    \begin{itemize}
        \item[] \includegraphics[width=1.0\textwidth]{ReusablePipeline.png}
    \end{itemize}
    \end{column}
  \end{columns}
\end{frame}


\begin{frame}{What is A Reproducible Analytical Pipeline?}
  \begin{columns}[T]
    \begin{column}{0.5\textwidth}

      \begin{itemize}[<+->]
        \item[] \only<1>{   RAP could be thought of as an approach to working }
        \item RAP is thus a robust process
        \item It is (\emph{quite}) automated
        \item It is (\emph{easily}) reproducible
        \item It minimizes the time to find and fix mistakes when they do occur
        \item It leads to fast processes\\ (see \href{https://github.com/Vanuatu-National-Statistics-Office/vnso-RAP-marketStats-materials}{Vanuatu Experience})
        \item It builds trust
      \end{itemize}
    \end{column}
    \begin{column}{0.5\textwidth}
    \begin{itemize}
        \item[] \includegraphics[width=1.0\textwidth]{ReusablePipeline.png}
    \end{itemize}
    \end{column}
  \end{columns}
\end{frame}

\begin{frame}{What does a RAP look like?}
 \begin{center}
  \begin{itemize}
        \only<1>{\includegraphics[width=0.8\textwidth]{Pipeline1.png} \\ Idealy, Input (website) and output (report) are linked}
        \only<2>{\includegraphics[width=0.8\textwidth]{Pipeline2.png} \\ Many steps are needed to create a report}
        \only<3>{\includegraphics[width=0.8\textwidth]{Pipeline3.png} \\ All steps should be linked in a structured process}
        \only<4>{\includegraphics[width=0.8\textwidth]{Pipeline4.png} \\ And only through code (Python, R), no copy/paste }
        \only<5>{\includegraphics[width=0.8\textwidth]{Pipeline6.png} \\ There may be side-products, but with explicit output-input links}
        \only<6>{\includegraphics[width=0.8\textwidth]{Pipeline7.png} \\ If needed, code can be updated (new versions)}
        \only<7>{\includegraphics[width=0.8\textwidth]{Pipeline8.png} \\ And comments added for each change}
        \only<8>{\includegraphics[width=0.8\textwidth]{Pipeline9.png} \\ Documentation on the process builds up with code changes}
        \only<9>{\includegraphics[width=0.8\textwidth]{Pipeline10.png} \\ Other contributors are welcome!}
    \end{itemize}
\end{center}
\end{frame}


\section{Principles}

\begin{frame}[<+->]
   \frametitle{RAP principles:}
   \pause
    \begin{itemize}
     \item Automation (\emph{as much as you can})
     \item[$\hookrightarrow$] Avoid manual work %(prone to errors, time consuming)
     \item Reusable (modular) code
     \item[$\hookrightarrow$] Build blocs, update blocs, change blocs, test blocs
     \item Transparency
     \item[$\hookrightarrow$] Show what you, do what you say
     \item Use open source tools
     \item[$\hookrightarrow$] Free, reusable, huge community % (+ chatGPT compatible)
     \item Version control
     \item[$\hookrightarrow$] Easy to track code, easy to share, easy to update, ...
     \item Good coding practices
     \item[$\hookrightarrow$] Write for humans, not for machines
     \item Testing
     \item Peer-review
    \end{itemize}
\end{frame}


\begin{frame}[<+->]
   \frametitle{RAP principles:}
   \pause
   \begin{itemize}[<+->]
     \item[] \begin{center}
             These principles translate into: \\ \vspace{0.5cm}
             \textbf{Good Practices }\\ + \\
             \textbf{Good Tools }
    \end{center}
     \item[$\hookrightarrow$] \vspace{2cm} We'll detail some of these practices and tools
   \end{itemize}

\end{frame}


\subsection{Good practices}

\begin{frame}
\frametitle{\textbf{Good practices:} Organize your work }
\textcolor{siap}{\textbf{Have a clear directory structure}}
 \pause
\begin{columns}[t]
 \begin{column}{0.4\textwidth}

    \begin{itemize}[<+->]
   \item Separate files into data, code, docs, etc.
   \item Make directories portable\\ (relative path)
    \end{itemize}
\end{column}
  \begin{column}{0.6\textwidth}
    \begin{center}
    \begin{itemize}
         \only<2>{ \includegraphics[width=0.5\textwidth]{WorkingDirectory.png} \\  }
         \only<2>{\hfill \tiny{ \textcolor{gris}{Example of a well-organized directory structure.}}\\ }
         \only<3-4>{\tiny{ \textbf{Usual}} \\ }
         \only<3-4>{ \tiny{Mydata <- read.csv("c://document/2024/RAPCourse/Data/TradeData.csv")}  \\ }
         \only<4>{\tiny{ \textbf{Better}}  \\}
         \only<4>{ \tiny{Mydata <- read.csv("../Data/TradeData.csv")}  }
    \end{itemize}
    \end{center}
  \end{column}
\end{columns}
\end{frame}


\begin{frame}
\frametitle{\textbf{Good practices:} Organize your work }
\textcolor{siap}{\textbf{Use naming conventions: } \\ }
For \textbf{files/code}\\
\begin{columns}[t]
 \begin{column}{0.3\textwidth}
    \begin{itemize}[<+->]
   \item Avoid  lazy names
   \item Meaningful files names
   \item Order of execution
    \end{itemize}
\end{column}
  \begin{column}{0.2\textwidth}
    \begin{itemize}
       \item[]
        \only<1-3>{ \small Usual \\ }
        \only<1-3>{ \small \texttt{prog1.R}\\
                    \texttt{prog2.R}\\
                    \texttt{Stat.R}\\
                    \texttt{progC.R}\\
                    \texttt{progP.R} \\ }
    \end{itemize}
  \end{column}
  \begin{column}{0.5\textwidth}
    \begin{itemize}
        \item[]
        \only<2>{\small Better \\ }
        \only<2>{ \small    \texttt{Scraping\_Data.R}\\
                            \texttt{Cleaning\_Data.R}\\
                            \texttt{Stats\_Tables.R}\\
                            \texttt{Classification.R}\\
                            \texttt{Price\_CPI.R} }
        \only<3>{\small Even better \\ }
        \only<3>{ \small
                    \texttt{01\_Scraping\_data.R}\\
                    \texttt{02\_Cleaning\_data.R}\\
                    \texttt{03\_Classification.R}\\
                    \texttt{04\_Stats\_Tables.R}\\
                    \texttt{04\_Price\_CPI.R} }

    \end{itemize}
  \end{column}
\end{columns}
\end{frame}

\begin{frame}
\frametitle{\textbf{Good practices:} Organize your work }
\textcolor{siap}{\textbf{Use naming conventions:} \\ }
For \textbf{outputs}
\begin{columns}[t]
 \begin{column}{0.3\textwidth}
    \begin{itemize}[<+->]
   \item Avoid numbering
   \item Explicit type of output
    \end{itemize}
\end{column}
  \begin{column}{0.2\textwidth}
    \begin{itemize}
    \item[]
        \only<1-2>{\small Usual \\ }
        \only<1-2>{\small \texttt{Table1.pdf} \\
                    \texttt{Table2.pdf} \\
                    \texttt{Graph.jpg} \\
                    \texttt{Model.csv} \\ }
    \end{itemize}
  \end{column}
  \begin{column}{0.5\textwidth}
    \begin{itemize}
    \item[]
        \only<2>{\small Better \\ }
        \only<2>{ \small  \texttt{Stat\_Desc\_Table.pdf} \\
                    \texttt{Price\_Stat\_Table.pdf}\\
                    \texttt{Dress\_Prices\_Graphic.jpg}\\
                    \texttt{All\_prices\_Results.csv} \\ }
    \end{itemize}
  \end{column}
\end{columns}
\end{frame}


\begin{frame}
\frametitle{\textbf{Good practices for automation} }
\textcolor{siap}{\textbf{Keep track of the workflow:} \\  }
 \begin{columns}[t]
 \begin{column}{0.5\textwidth}
    \begin{itemize}[<+->]
     \item Cut and paste should be avoided
     \item Every step of the process is coded
     \item Manage (and draw) the workflow
   \end{itemize}
  \end{column}
 \begin{column}{0.5\textwidth}
    \begin{itemize}
        \item[]
        \only<1-3>{ \includegraphics[width=1.0\textwidth]{Workflow.png} \\ }
        \only<1-3>{\hfill \tiny{ \textcolor{gris}{Example of a simple workflow.}} }
    \end{itemize}
  \end{column}
\end{columns}
\end{frame}


\begin{frame}
\frametitle{\textbf{Good practices for  automation} }
\textcolor{siap}{\textbf{Keep track of the workflow:} \\  }

Here is a workflow of a  web scraping process \\
\begin{center}
\href{https://github.com/sergegoussev/ESCAP_RAP_class/blob/main/docs/images/ads-process-overview-high-level-overview.drawio.svg}
{\includegraphics[width=0.9\textwidth]{ESCAPWebScrappingPipeline.png}} \\
\hfill \tiny{ \textcolor{gris}{Created by Serge Goussev.}}
\end{center}

\end{frame}

\begin{frame}
\frametitle{\textbf{Very good practices} }
\textcolor{siap}{\textbf{Use a version control system  (Git/GitHub)}} \\
\vspace{0.5cm}
\begin{center}
 \includegraphics[width=0.7\textwidth]{FileHistory.PNG} \\

\large{\textbf{ More on Version Control later }}

\end{center}
\end{frame}

\subsection{Good coding practices}

\begin{frame}[<+->]
\frametitle{\textbf{Good coding practices: Code for others} }
\textcolor{siap}{\textbf{Program with style:}}
          \begin{itemize}
             \item[]  \only<1>{Use \emph{literate programming} }
             \only<1>{ \begin{center}
                    ``\textit{Let us concentrate rather on explaining to humans \\ what we want the computer to do}'' \\
                    \hfill \textcolor{gris}{ D. Knuth (1984)}
             \end{center}}

             \only<2>{ \begin{center}
                    ``\textit{($\cdots$) code is read much more often than it is written}''  \\
                    \hfill \textcolor{gris}{  \href{https://peps.python.org/pep-0008/}{Guido van Rossum (2013 -PEP8)}}
             \end{center}}
         \end{itemize}
\end{frame}


\begin{frame}
\frametitle{\textbf{Good coding practices: Code for others} }
\textcolor{siap}{\textbf{Program with style}} \\
\begin{columns}[t]
 \begin{column}{0.3\textwidth}
    \begin{itemize}[<+->]
   \item Avoid ambiguities
   \item Avoid changing units
    \end{itemize}
\end{column}
  \begin{column}{0.7\textwidth}
    \begin{itemize}
    \item[]
        \only<1>{\scriptsize Usual \\ }
        \only<1>{\scriptsize \texttt{sex <- ifelse(gender == "1001", 1, 2)} \\}
        \only<1>{\small Better \\ }
        \only<1>{\scriptsize \texttt{female <- ifelse(gender == "1001",1,0)} \\
                                 \texttt{     male <- ifelse(gender != "1001",1,0)} \\ }
         \only<2-4>{\scriptsize Usual \\ }
        \only<2-4>{\scriptsize \texttt{gdp <- gdp/118.722 } \\
                                \texttt{ } \\ }
        \only<3>{\scriptsize Better \\ }
        \only<3>{\scriptsize \texttt{gdp\_US <- gdp / 118.722}  }
        \only<4>{\scriptsize Even better \\ }
        \only<4>{\scriptsize \texttt{US\_Vanu\_exch\_rate <- 118.722} \\
                                 \texttt{gdp\_US <- gdp / US\_Vanu\_exch\_rate}  }

    \end{itemize}
  \end{column}
\end{columns}
\end{frame}

%\end{document}

%%%%%%
\subsection{Modularity}
\begin{frame}
\frametitle{\textbf{Good practices: Modularity} }

\textcolor{siap}{\textbf{Create reusable objects}}
\begin{columns}[t]
 \begin{column}{0.2\textwidth}
    \begin{itemize}[<+->]
       \item { \scriptsize Store values}
       \item[]{ \scriptsize Avoid repetitions}
       \item[]  % Empty item
       \item { \scriptsize Use functions}
       \item { \scriptsize Use independent blocks}
    \end{itemize}
\end{column}
  \begin{column}{0.8\textwidth}
    \begin{itemize}
         \item[]
        \only<1-2>{\scriptsize Usual \\ }
        \only<1-2>{\scriptsize \texttt{Current\_Data <- subset(Mydata, year  ==2023)} \\}
        \only<2>{\scriptsize Better \\ }
        \only<2>{\scriptsize \texttt{Current\_year <- 2023 \\
                        \texttt{Current\_Data <- subset(Mydata,} \\
                       \hfill \texttt{year == Current\_year)}} \\ }
        \only<3>{\scriptsize Usual \\ }
        \only<3>{\scriptsize \texttt{data <- Mydata[Mydata\$export == "Beef", ]} \\
                 \scriptsize \texttt{plot(data\$Year, data\$Value, } \\
                 \hfill \scriptsize \texttt{ main =  "Export for Beef") \\ }
                 \scriptsize \texttt{  } \\ }
        \only<3>{\scriptsize \texttt{data <- Mydata[Mydata\$export == "Kava", ]} \\
                \scriptsize \texttt{plot(data\$Year, data\$Value, } \\
                 \hfill \scriptsize \texttt{ main =  "Export for Kava") \\ }
                 \texttt{ ... } \\ }
        \only<4>{\scriptsize Better \\ }
        \only<4>{\scriptsize \texttt{ type <- "Beef"} \\
                                \texttt{ } \\
                                \texttt{  Mydata \%>\% } \\
                                \texttt{    filter(exports == type) \%>\%}\\
                                \texttt{    ggplot() + }\\
                                \texttt{    aes(x = Year, y = Value) +}\\
                                \texttt{    geom\_point() + }\\
                                \texttt{    ggtitle(paste("Export for ", type)) }\\
                                \texttt{ } \\ }
         \only<5>{\scriptsize Even better \\ }
         \only<5>{\scriptsize \texttt{Exports\_graphic <- function(type) \{ } \\
                                \texttt{  Mydata \%>\% } \\
                                \texttt{    filter(exports == type) \%>\%}\\
                                \texttt{    ggplot() + }\\
                                \texttt{    aes(x = Year, y = Value) +}\\
                                \texttt{    geom\_point() + }\\
                                \texttt{    ggtitle(paste("Export for ", type)) }\\
                                \texttt{\} }\\
                                \texttt{ } \\
                                \texttt{Exports\_graphic("Beef")} \\
                                \texttt{Exports\_graphic("Kava")   }}

    \end{itemize}
  \end{column}
\end{columns}
\end{frame}

 % Could we integrate https://nhsdigital.github.io/rap-community-of-practice/images/tight-loose-coupling.png into this slide?

%\begin{frame}
%\frametitle{Other principles}
%    \begin{itemize}[<+->]
%        \item Discuss with colleagues that may use your work
%        \item Automatize as much as you can
%        \item[$\hookrightarrow$] Reduces your brain's memory burden
%        \item There are easy steps everybody can do
%        \item[$\hookrightarrow$] Write small programs, one for each task
%        \item Use open source program
%        \item[$\hookrightarrow$] Easier to share, easier to automatize
%        \item[$\hookrightarrow$] Also cost-effective
%        \item Test your work regularly:
%        \only<11>{\begin{center}
%            \emph{``Do what has been said, say what has been done, and \\ check that what has been said has really been done !''}
%         \end{center}}
%        \only<12>{\begin{center}
%            \emph{``\textbf{Code} what has been said, say what has been \textbf{coded}, and \\ check that what has been said has really been \textbf{coded}  !''}
%         \end{center}}
%    \end{itemize}
%\end{frame}


\section{Version Control}

\begin{frame} % Cover slide
\frametitle{Version Control keeps tracks of your work}
Tracking three \textbf{W} questions:
\begin{columns}[t]
 \begin{column}{0.5\textwidth}
 \begin{itemize}[<+->]
 \item[] \textcolor{siap}{\textbf{W}}hat changes?
 \item[] \textcolor{siap}{\textbf{W}}ho made the changes?
 \item[] \textcolor{siap}{\textbf{W}}hen were the changes made?
 \end{itemize}
 \end{column}
  \begin{column}{0.5\textwidth}
    \begin{center}
    \begin{itemize}
        \only<1-4>{ \includegraphics[width=0.95\textwidth]{FileHistory.PNG} \\  }
       \only<1-4>{\hfill  \textcolor{gris}{\tiny{Source: \href{https://the-turing-way.netlify.app/reproducible-research/vcs.html}{The Turing Way project}}}}
    \end{itemize}
    \end{center}
  \end{column}
\end{columns}
\end{frame}



\begin{frame}{Transparency, Accountability \& Reproducibility}
    \begin{itemize}[<+->]
        \item Version control provides a detailed history of changes
        \item Each modification is attributed to a specific user
        \item Promotes accountability, transparency \&  reproducibility
    \end{itemize}
\end{frame}

\subsection{File evolution}

\begin{frame}{File evolution  \textcolor{brique}{without Version Control}  }
\begin{center}
\begin{itemize}
   \only<1> {\includegraphics[width = 1.0\textwidth]{FileChange1a.png} \\ }
   \only<2> {\includegraphics[width = 1.0\textwidth]{FileChange2a.png} \\ }
   \only<3> {\includegraphics[width = 1.0\textwidth]{FileChange3a.png} \\ }
   \only<4> {\includegraphics[width = 1.0\textwidth]{FileChange4a.png} \\ }
   \only<5> {\includegraphics[width = 1.0\textwidth]{FileChange5a.png} \\ }
   \only<6> {\includegraphics[width = 1.0\textwidth]{FileChange6a.png} \\ }
   \only<7> {\includegraphics[width = 1.0\textwidth]{FileChange7a.png} \\ }
\end{itemize}
\end{center}
\end{frame}

\begin{frame}{File evolution  \textcolor{brique}{without Version Control}  }
Usual ways to keep track of changes:
\pause
\begin{columns}[t]
\begin{column}{0.6\textwidth}
\begin{itemize}[<+->]
        \item New file after each change
        \item[$\hookrightarrow$] Need to open each file to see the change
        \item[$\hookrightarrow$] Names have to be explicit
        \item Only the last file with lots of comments
        \item Not fulfilling the 3 \textcolor{siap}{\textbf{W}}...

    \end{itemize}
 \end{column}
  \begin{column}{0.4\textwidth}
    \begin{center}
    \begin{itemize}
        \only<1-4>{ \includegraphics[width=1.0\textwidth]{FileLifeAll.png} \\  }
        \only<5-6>{ \includegraphics[width=0.95\textwidth]{FileLifeFinalFinal.png} \\  }
       % \only<7>{ \includegraphics[width=1.0\textwidth]{FileLifeAll.png} \\  }
       % \only<7>{\hfill \includegraphics[width=0.5\textwidth]{FileLifeFinalFinal.png} \\  }

    \end{itemize}
    \end{center}
  \end{column}
\end{columns}
\end{frame}

\subsection{Commits}

\begin{frame}{File evolution  \textcolor{brique}{with Version Control}  }
You can see exactly what has been going on!
\begin{center}
\begin{itemize}
   \only<1> {\includegraphics[width = 0.8\textwidth]{GitHubDiff.png} \\ }
\end{itemize}
\end{center}
\end{frame}



\begin{frame}{File evolution  \textcolor{brique}{with Version Control}  }
Record a message (\emph{commit})  for each change!
\begin{center}
\begin{itemize}
   \only<1> {\includegraphics[width = 1.0\textwidth]{FileLife1.png} \\ }
   \only<2> {\includegraphics[width = 1.0\textwidth]{FileLife2.png} \\ }
   \only<3> {\includegraphics[width = 1.0\textwidth]{FileLife3.png} \\ }
   \only<4> {\includegraphics[width = 1.0\textwidth]{FileLife4.png} \\ }
   \only<5> {\includegraphics[width = 1.0\textwidth]{FileLife5.png} \\ }
   \only<6> {\includegraphics[width = 1.0\textwidth]{FileLife6.png} \\ }
   \only<7> {\includegraphics[width = 1.0\textwidth]{FileLife7.png} \\ }
\end{itemize}
\end{center}
\end{frame}

\begin{frame}{The history of the file is recorded!}
\begin{center}
\begin{itemize}
   \only<1-2> {Each version is documented (with \emph{commits}) \\ }
   \only<1-2> {\includegraphics[width = 1.0\textwidth]{FileLifeHistoryEnd.png} \\ }
   \only<3-4> {Each version embeds the full history!  }
   \only<3-4> {\includegraphics[width = 1.0\textwidth]{FileLifeHistoryFull.png} \\ }
\end{itemize}
\end{center}
\end{frame}

\begin{frame}{Going back (\emph{revert}) is possible}
\begin{center}
\begin{itemize}
   \only<1-2> {It is possible to review previous version... \\ }
   \only<1-2> {\includegraphics[width = 1.0\textwidth]{FileLifeBack.png} \\ }
   \only<3-4> {...to compare the changes...  }
   \only<3-4> {\includegraphics[width = 1.0\textwidth]{FileLifeDiff.png} \\ }
   \only<5-6> {... and to revert  to a  previous version... }
   \only<5-6> {\includegraphics[width = 1.0\textwidth]{FileLifeRevert.png} \\ }
   \only<7-8> {... or \emph{restart} from there as if nothing happened }
   \only<7-8> {\includegraphics[width = 1.0\textwidth]{FileLifeRevert2.png} \\ }

\end{itemize}
\end{center}
\end{frame}


\begin{frame}{Version Control in a Nutshell }
A Version Control systems:
\begin{columns}[t]
\begin{column}{0.8\textwidth}
\begin{itemize}[<+->]
    \item Keeps track of all changes
    \item Allows you to ignore anything you don't want to version control (such as internal data) in the (\emph{.gitignore})
    \item Allows reviewing stages of development
    \item Allow collaborating on projects
    \item Comes with different tools (Git, GitHub, GitLab, etc..)!
    \item[$\hookrightarrow$]  GitLab can be set up on an internal NSO network.
    \item Backups your work
  \end{itemize}
 \end{column}
  \begin{column}{0.2\textwidth}
    \begin{center}
    \begin{itemize}
        \only<1-6>{\includegraphics[width=0.5\textwidth]{github-mark.png} \\ \vspace{0.5cm}
                    \includegraphics[width=0.5\textwidth]{gitlab-logo.png} \\ }
    \end{itemize}
    \end{center}
  \end{column}
\end{columns}
\end{frame}



\section{Takeaways}

\begin{frame}{Towards Good Practices?  }
\pause
\begin{itemize}[<+->]
    \item Good practices starts with our own practices
    \item[$\hookrightarrow$]  KISS: Keep it Simple, Stupid
    \item There many levels of RAP (a full spectrum)
    \item[$\hookrightarrow$] Start small, be an advocate for others
    \item[$\hookrightarrow$] Increase complexity when ready
    \item Building a RAP is a collective process \\
        %\item[$\hookrightarrow$] Sharing tools, processes \& feedback \\
        \includegraphics[width=0.5\textwidth]{DecolonisingKnowledge.jpg}
\end{itemize}
\end{frame}


\section{Resources}
\begin{frame}{Useful resources}
  \begin{itemize}
  \item  \href{https://nhsdigital.github.io/rap-community-of-practice/introduction_to_RAP/what_is_RAP/}{NHS Community of Practice}
  % \item  \href{https://github.com/sergegoussev/ESCAP_RAP_class/tree/main}{This course website} (created by Serge Goussev)
   \item \href{https://github.com/Vanuatu-National-Statistics-Office/vnso-RAP-marketStats-materials}{Vanuatu Bureau of Statistics implementation of RAP}
   \item \href{https://www.unsiap.or.jp/on_line/2024/RAP/flyer-RAP-Self-paced.pdf}{SIAP's (free) online RAP course}
    \item The UK government RAP \href{https://ukgovdatascience.github.io/rap-website/index.html}{website}.
    \item UK best practice \href{https://gss.civilservice.gov.uk/policy-store/quality-statistics-in-government/\#reproducible-analytical-pipelines-rap-}{documentation}.
    \item A free RAP \href{https://www.udemy.com/course/reproducible-analytical-pipelines/}{course} to teach you all you need to know.
    \item How the Data Science Campus sets its coding \href{https://datasciencecampus.github.io/coding-standards/}{standards}.
    \item A new open-source \href{https://the-turing-way.netlify.com}{book} from the Alan Turing institute setting out how to do reproducible data science.
  \end{itemize}
\end{frame}




\end{document}


