\documentclass[xcolor=x11names,compress]{beamer}
%% General document
\usepackage{graphicx, subfig}
%% Beamer Layout
\useoutertheme[subsection=false,shadow]{miniframes}
\useinnertheme{default}
\usefonttheme{serif}
\usepackage{palatino}

%%%%%%% Mes Packages %%%%%%%%%%%%%%%%
%\usepackage[french]{babel}
\usepackage[T1]{fontenc}
\usepackage{color}
\usepackage{xcolor}
\usepackage{dsfont} % Pour indicatrice
\usepackage{url}
\usepackage{multirow}
\usepackage[normalem]{ulem}   % For strike out text

% Natbib for clean bibliography
\usepackage[comma,authoryear]{natbib}

%remove the icon
\setbeamertemplate{bibliography item}{}

%remove line breaks
\setbeamertemplate{bibliography entry title}{}
\setbeamertemplate{bibliography entry location}{}
\setbeamertemplate{bibliography entry note}{}

%% ------ MEs couleurs --------
\definecolor{vert}{rgb}{0.1,0.7,0.2}
\definecolor{brique}{rgb}{0.7,0.16,0.16}
\definecolor{gris}{rgb}{0.7, 0.75, 0.71}
\definecolor{twitterblue}{rgb}{0, 0.42, 0.58}
\definecolor{airforceblue}{rgb}{0.36, 0.54, 0.66}
\definecolor{siap}{RGB}{3,133, 200}


%%%%%%%%%%%%%%%%% BEAMER PACKAGE %%%%%%%

\setbeamercolor{itemize item}{fg=siap}
%\setbeamercolor{itemize subitem}{fg=blue}
%\setbeamercolor{itemize subsubitem}{fg=cyan}

\setbeamerfont{title like}{shape=\scshape}
\setbeamerfont{frametitle}{shape=\scshape}

\setbeamercolor*{lower separation line head}{bg=DeepSkyBlue4}
\setbeamercolor*{normal text}{fg=black,bg=white}
\setbeamercolor*{alerted text}{fg=siap}
\setbeamercolor*{example text}{fg=black}
\setbeamercolor*{structure}{fg=black}
\setbeamercolor*{palette tertiary}{fg=black,bg=black!10}
\setbeamercolor*{palette quaternary}{fg=black,bg=black!10}

% Set the header color to SIAP's color
\setbeamercolor*{frametitle}{fg=siap}

%remove navigation symbols
\setbeamertemplate{navigation symbols}{}

\renewcommand{\(}{\begin{columns}}
\renewcommand{\)}{\end{columns}}
\newcommand{\<}[1]{\begin{column}{#1}}
\renewcommand{\>}{\end{column}}

%% Add footer with logo
\setbeamertemplate{footline}{%
  \begin{beamercolorbox}[wd=\paperwidth,ht=2.5ex,dp=1.125ex,%
    leftskip=.3cm,rightskip=.3cm plus1fil]{author in head/foot}
    \includegraphics[height=5ex]{SIAP_logo_Big.png}\hfill
    \insertshortauthor\hfill\insertshorttitle\hfill  \textcolor{siap}{\textit{\insertframenumber}}
  \end{beamercolorbox}%
}
% Path for the graphs
\graphicspath{
{Graphics/}{../../../../Visualisation/Presentations/Graphics/}
{../../Visualisation/Presentations/Graphics/}
{c:/Chris/Visualisation/Presentations/Graphics/Logos/}
{c:/Gitmain/MLCourse/UNML/Module4/M4-0-SimpleTrees_files/figure-html/}
{c:/Gitmain/MLCourse/UNML/Module4/M4-1-DecisionTrees_files/figure-html/}
{c:/Gitmain/MLCourse/UNML/Module2/M2-1-SimpleClassification_files/figure-html/}
{c:/Gitmain/MLCourse/UNML/Module4/M4-2-RandomForest_files/figure-html/}
{c:/Gitmain/MLCourse/UNML/Module5/M5-1-SVM_files/figure-html/}
{c:/Chris/UN-ESCAP/MyCourses2022/MLOS2022/Slides/Graphics/}
}

%remove navigation symbols
\setbeamertemplate{navigation symbols}{}


% Natbib for clean bibliography
\usepackage[comma,authoryear]{natbib}

%%%% Use it or not %%%%

\title{\textcolor{siap}{Machine Learning for Official Statistics in Asia-Pacific countries \\ \vspace{0.5cm} }}

\subtitle{\textcolor{brique}{\Large{Principles of Support Vector Machine (SVM)}}}
\author{\textcolor{siap}{Christophe Bontemps}}
\institute{ \includegraphics[height=9ex]{SIAP_logo_Big.png}}
\date{}

\begin{document}

\begin{frame}
  \titlepage
\end{frame}


\section{Introduction}

\begin{frame} % Cover slide
\frametitle{\textcolor{brique}{[ Understanding  SVM   ]}}
\textit{Support Vector Machine} (SVM) is a classifier\\
\pause
\begin{itemize}[<+->]
  \item One of the best "out of the box" classifier
  \item Lots of refinements
  \item Based on two very different ideas
  \begin{itemize}[<+->]
    \item Dimension augmentation
    \item Maximal margin classifier
  \end{itemize}
  \item Technical details require advanced mathematics
  \item[$\hookrightarrow$] Focus on seminal ideas and intuitions
\end{itemize}
\end{frame}

\section{The VC dimension}

\begin{frame} % Cover slide
\frametitle{\textcolor{brique}{[ The  Vapnik–Chervonenkis dimension ]}}
\pause
\begin{itemize}[<+->]
    \item The \textit{Vapnik–Chervonenkis} (VC) dimension measures the complexity that can be learned by a classifier
    \item The VC dimension is defined as the maximum  number of points allowing a separation by a linear classifier
    \item[$\hookrightarrow$]  $ VC(f)$ = \# points always separable in the current space
    \item Useful to measure the complexity of a particular model in a given space
\end{itemize}
\end{frame}

\begin{frame} % Cover slide
\frametitle{\textcolor{brique}{[ The  Vapnik–Chervonenkis dimension]}}
\begin{itemize}
   \item[]
   \only<2-5>{In  1-D, there are 4 situation with 2 points: \\ }
   \only<2> {\includegraphics[width = 0.8\textwidth]{VC1D-OO.png} \\ }
   \only<2> {Both orange }
   \only<3> {\includegraphics[width = 0.8\textwidth]{VC1D-BB.png} \\ }
   \only<3> {Both blue}
   \only<4> {\includegraphics[width = 0.8\textwidth]{VC1D-OB.png} \\ }
   \only<4> {One orange, one blue}
   \only<5> {\includegraphics[width = 0.8\textwidth]{VC1D-BO.png} \\ }
   \only<5> {One blue, one orange}
   \only<6>{ What happens with 3 points?}
   \only<6> {\includegraphics[width = 0.8\textwidth]{VC1D-OBO.png}  \\ }
   \only<6>{\emph{A situation where no linear classifier works!}  \\
            $\hookrightarrow$ $VC(f_{lin})$ = 2}
\end{itemize}
\end{frame}

\begin{frame} % Cover slide
\frametitle{\textcolor{brique}{[ The  Vapnik–Chervonenkis dimension]}}
\pause
\begin{itemize}
  \item[]
  \only<2-10>{In 2-D, there are  $2^3 = 8$ different situations with 3 points: \\  \vspace{0.8cm}  }
   \only<2> {\includegraphics[width = 0.7\textwidth]{VC2D-OOO.png} \\ }
   \only<3> {\includegraphics[width = 0.7\textwidth]{VC2D-BOO.png} \\ }
   \only<4> {\includegraphics[width = 0.7\textwidth]{VC2D-OBO.png} \\ }
   \only<5> {\includegraphics[width = 0.7\textwidth]{VC2D-OOB.png}  \\ }
   \only<6> {\includegraphics[width = 0.7\textwidth]{VC2D-BBO.png} \\ }
   \only<7> {\includegraphics[width = 0.7\textwidth]{VC2D-BOB.png} \\ }
   \only<8> {\includegraphics[width = 0.7\textwidth]{VC2D-OBB.png} \\ }
   \only<9> {\includegraphics[width = 0.7\textwidth]{VC2D-BBB.png}  \\ }
   \only<10> { \includegraphics[width = 0.2\textwidth]{VC2D-OOO.png}
               \includegraphics[width = 0.2\textwidth]{VC2D-BOO.png}
               \includegraphics[width = 0.2\textwidth]{VC2D-OBO.png}
               \includegraphics[width = 0.2\textwidth]{VC2D-OOB.png}\\
               \vspace{1cm}
               \includegraphics[width = 0.2\textwidth]{VC2D-BBO.png}
               \includegraphics[width = 0.2\textwidth]{VC2D-BOB.png}
               \includegraphics[width = 0.2\textwidth]{VC2D-OBB.png}
               \includegraphics[width = 0.2\textwidth]{VC2D-BBB.png} \\  }

   \only<11>{ What happens with 4 points?}
   \only<11> {\includegraphics[width = 0.7\textwidth]{VC2D-OBBO.png}  \\ }
   \only<11> {\emph{A situation where no linear classifier works!} \\
            $\hookrightarrow$ $VC(f_{lin})$ = 3}
\end{itemize}
\end{frame}


\begin{frame} % Cover slide
\frametitle{\textcolor{brique}{[ Interpretation of the VC dimension ]}}
\pause
\begin{itemize}[<+->]
    \item $VC(f)$  measures the complexity of the classifier $f$, linear or not
    \item A highly non-linear classifier will always be able to separate a high number of points
    \item[] It can be shown that:
    \begin{itemize}[<+->]
    \item Classifiers with a \textbf{high} VC dimension will be better at learning and fitting the training data set
    \item[$\hookrightarrow$]  \textbf{High} risk of over-fitting
    \item Classifiers with a \textbf{low} VC dimension make more errors on the training data set, but may be better in prediction
    \item[$\hookrightarrow$]  \textbf{Low} risk of over-fitting
    \end{itemize}
\item New version of the \textbf{Bias-Variance} trade off!
\end{itemize}
\end{frame}

\begin{frame} % Cover slide
\frametitle{\textcolor{brique}{[ Implications of the VC dimension ]}}
\pause
\begin{itemize}[<+->]
    \item The  Vapnik–Chervonenkis inequality (\emph{simplified}):
    $$
        \epsilon_{\;validation}(f)  \leq   \epsilon_{\;train}(f)  + G( VC(f) , m)
    $$
    \item[] Where:
    \begin{itemize}[<+->]
        \item[-] $\epsilon(\cdot)$ is the error rate,
        \item[-] $m$ is the nb of training sets and
        \item[-]  $G(\cdot)$  a function \textbf{decreasing }with  $m$ (obviously)  % the more you train, the closest the valisdation error to the  trianing arreor
        \item[]  \hspace{2.45cm}  \textbf{increasing} with $VC$ !! % The higher $VC$ the greater the difference between  \eps_{validation}(f)  and  \eps_{train}(f)
    \end{itemize}
    \item[$\hookrightarrow$] The higher $VC(f)$ the greater the difference between performance on the \textit{training} and \textit{validation} sample
    \item[$\hookrightarrow$] Better to use simple models (low VC)
\end{itemize}
\end{frame}

\section{Dimension augmentation}

\begin{frame}
\frametitle{\textcolor{brique}{[ Linear Boundaries in a simple example ]}}
\pause
\begin{itemize}[<+->]
    \item[] Take the following hypothetical situation
     \only<2> {\includegraphics[width = 0.8\textwidth]{initial-1.png} \\ }
     \only<3> {\includegraphics[width = 0.8\textwidth]{initial-2.png} \\ }
    \item[$\hookrightarrow$] It is easy to \textit{separate} the two classes with a linear  classifier
\end{itemize}
\end{frame}

\begin{frame}
\frametitle{\textcolor{brique}{[ Linear Boundaries in a simple example ]}}
\pause
\begin{itemize}[<+->]
    \item[] But in this  situation?
     \only<2-3> {\includegraphics[width = 0.8\textwidth]{overlap-1.png} \\ }
    \item[$\hookrightarrow$] There is no way to  \textit{separate} the two classes with a linear  classifier!!
\end{itemize}
\end{frame}


\begin{frame} % Cover slide
\frametitle{\textcolor{brique}{[ Linear Boundaries in a simple example ]}}
\pause
\begin{itemize}[<+->]
  \item[] \begin{center} \huge Unless... \end{center}
\end{itemize}
\end{frame}

\begin{frame}
\frametitle{\textcolor{brique}{[ Dimension augmentation]}}
\pause
\begin{itemize}[<+->]
    \item[] Unless  one adds a third dimension!
     \only<2> {\includegraphics[width = 0.8\textwidth]{augmented-1.png} \\ }
     \only<3> {\includegraphics[width = 0.8\textwidth]{augmented-2.png} \\ }
    \item[$\hookrightarrow$] A  linear classifier can \textit{separate} the two classes  in an \textit{augmented} space (third dimension)
\end{itemize}
\end{frame}

\section{Soft margin}

\begin{frame}
\frametitle{\textcolor{brique}{[ Maximal Margin Classifier ]}}
\pause
\begin{itemize}[<+->]
    \item[] The maximum margin classifier is defined as:
     \only<2> {\includegraphics[width = 0.8\textwidth]{augmented-2.png} \\ }
     \only<3> {\includegraphics[width = 0.8\textwidth]{marginal-1.png} \\ }
    \item[$\hookrightarrow$] $M$ is the margin.
\end{itemize}
\end{frame}


\begin{frame}
\frametitle{\textcolor{brique}{[ Maximal Margin Classifier ]}}
\pause
\begin{itemize}[<+->]
    \item[] The optimization problem to find this margin is :\\
    \only<2> {\emph{with $(\beta_0 +x'\beta)$ defining the boundary (line or hyperplane) }\\}
    \only<2> {\begin{eqnarray*}
                Max_{\beta_0, \beta} & M & \\
                subject\; to:\; y_i(\beta_0 +x' \beta) &\geq &  M
               \end{eqnarray*}
               }
    \only<2> {and $y_i = 1 \; or -1$ depending on the class of observation $i$ }
    \only<3> {$\hookrightarrow$ This can be relaxed  \\ }
    \only<3> {\begin{eqnarray*}
                Max_{\beta_0, \beta} & M & \\
                subject\; to:\; y_i(\beta_0 +x' \beta) &\geq &  M - \xi
               \end{eqnarray*}
               }
   \only<3> {where $\xi$  is the \textit{slack} parameter}
\end{itemize}
\end{frame}


\begin{frame}
\frametitle{\textcolor{brique}{[ Soft margin ]}}
Introducing \textit{soft margins} is one idea implemented in SVM
\pause
\begin{itemize}[<+->]
    \item[] The slack parameter will play an important role
    \item A small \textit{slack} allows few exceptions
    \item[$\hookrightarrow$] Boundary (classifier) sensitive to outliers
    \item A large \textit{slack} allows lots of errors and misclassifications
    \item[] The optimization problem  is equivalent to:
    \begin{eqnarray}
                & Min_{\beta_0, \beta} &  \frac{1}{2} \| \beta \|^2 + C \sum_{i=1}^N \xi_i  \\
        \nonumber       & subject\; to:\; &  y_i(\beta_0 +x'\beta) \geq  M - \xi_i
    \end{eqnarray}
    \item[$\hookrightarrow$] SVM is a trade-off between maximal $M$ and cost $C$ of errors due to $\xi$
\end{itemize}
\end{frame}


\section{Kernel trick}


\begin{frame}
\frametitle{\textcolor{brique}{[ The \textit{kernel trick} ]}}
The previous optimization method applies to an (\textit{augmented}) space with separable classes
\pause
\begin{itemize}[<+->]
    \item How do we determine this \textit{augmented} space?
    \item[] How to find $\phi(\cdot)$ so that $X_3 = \phi(X_1, X_2)$  separates classes?
    \item[] \includegraphics[width = 0.6\textwidth]{marginal-1.png}
    \item[$\hookrightarrow$] The \emph{kernel trick!}
\end{itemize}
\end{frame}


\begin{frame}
\frametitle{\textcolor{brique}{[ The \textit{kernel trick} ]}}
Finding $\phi(\cdot)$ is difficult and sometimes infeasible!
\pause
\begin{itemize}[<+->]
   \item One can rewrite the optimization problem
    \item[$\hookrightarrow$] Transform the $Xs$ through kernels $K(\cdot)$
    \item A \emph{kernel} quantifies the similarity of two observations (\textit{inner product})
    \item The optimization can be written \textbf{only} in terms of $K(X_i, X_j)$!
    \item[$\hookrightarrow$] The solution is non linear in the  original space
    % \item[]\includegraphics[width = 0.6\textwidth]{overlap-1.png}
\end{itemize}
\end{frame}


\begin{frame}
\frametitle{\textcolor{brique}{[ Hyperparameters in SVM ]}}
Several hyperparameters have to be selected:
\pause
\begin{itemize}[<+->]
   \item The slack parameter
   \item The type of kernel (\textit{linear, radial, sigmoid}, ..)
   \item In an ML framework,  one trains the SVM on several samples
   \begin{itemize}[<+->]
        \item Number of CV sample should be reduced (100)
        \item Linear kernel are less computationally intensive
    \end{itemize}
    \item Proceed with parsimony and increase complexity
\end{itemize}
\end{frame}

\begin{frame} % Cover slide
\frametitle{\textcolor{brique}{[Quiz time]}}
\pause
\begin{itemize}[<+->]
  \item[]
\end{itemize}
\end{frame}


\section{Wrap-up}

\begin{frame} % Cover slide
\frametitle{\textcolor{brique}{[Takeaways]}}
SVM is a complex but popular technique
\begin{itemize}[<+->]
\item Based on several clever ideas:
\begin{itemize}[<+->]
    \item Dimension augmentation
    \item Maximal margin classifier
    \item[$\hookrightarrow$] Soft margins with a \textit{slack} parameter
    \item A trick to avoid costly features transformations
    \item[$\hookrightarrow$] The \textit{Kernel Trick}
    \end{itemize}
\item Due to the VC theory, SVM use simple models
\item SVM requires optimization and CPU
\item Implemented in many software!
\end{itemize}
\end{frame}


\end{document}
%%%%%%%%%%%%%%% Last Slide %%%%%%%%%%%%%%%%

\begin{frame}[allowframebreaks]%in case more than 1 slide needed
\frametitle{References}
    {\footnotesize
    %\bibliographystyle{authordate1}
    \bibliographystyle{apalike}
    \bibliography{../../../Visualisation/Visu}
    }
\end{frame}
\end{document}

%\bibliographystyle{authordate1}
%\bibliography{c:/Chris/Visualisation/Visu}
%\end{frame}
