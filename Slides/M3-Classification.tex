\documentclass[xcolor=x11names,compress]{beamer}
%% General document
\usepackage{graphicx, subfig}
%% Beamer Layout
\useoutertheme[subsection=false,shadow]{miniframes}
\useinnertheme{default}
\usefonttheme{serif}
\usepackage{palatino}

%%%%%%% Mes Packages %%%%%%%%%%%%%%%%
%\usepackage[french]{babel}
\usepackage[T1]{fontenc}
\usepackage{color}
\usepackage{xcolor}
\usepackage{dsfont} % Pour indicatrice
\usepackage{url}
\usepackage{multirow}
\usepackage[normalem]{ulem}   % For strike out text

% Natbib for clean bibliography
\usepackage[comma,authoryear]{natbib}

%remove the icon
\setbeamertemplate{bibliography item}{}

%remove line breaks
\setbeamertemplate{bibliography entry title}{}
\setbeamertemplate{bibliography entry location}{}
\setbeamertemplate{bibliography entry note}{}

%% ------ MEs couleurs --------
\definecolor{vert}{rgb}{0.1,0.7,0.2}
\definecolor{brique}{rgb}{0.7,0.16,0.16}
\definecolor{gris}{rgb}{0.7, 0.75, 0.71}
\definecolor{twitterblue}{rgb}{0, 0.42, 0.58}
\definecolor{airforceblue}{rgb}{0.36, 0.54, 0.66}
\definecolor{siap}{RGB}{3,133, 200}


%%%%%%%%%%%%%%%%% BEAMER PACKAGE %%%%%%%

\setbeamercolor{itemize item}{fg=siap}
%\setbeamercolor{itemize subitem}{fg=blue}
%\setbeamercolor{itemize subsubitem}{fg=cyan}

\setbeamerfont{title like}{shape=\scshape}
\setbeamerfont{frametitle}{shape=\scshape}

\setbeamercolor*{lower separation line head}{bg=DeepSkyBlue4}
\setbeamercolor*{normal text}{fg=black,bg=white}
\setbeamercolor*{alerted text}{fg=siap}
\setbeamercolor*{example text}{fg=black}
\setbeamercolor*{structure}{fg=black}
\setbeamercolor*{palette tertiary}{fg=black,bg=black!10}
\setbeamercolor*{palette quaternary}{fg=black,bg=black!10}

% Set the header color to SIAP's color
\setbeamercolor*{frametitle}{fg=siap}

%remove navigation symbols
\setbeamertemplate{navigation symbols}{}

\renewcommand{\(}{\begin{columns}}
\renewcommand{\)}{\end{columns}}
\newcommand{\<}[1]{\begin{column}{#1}}
\renewcommand{\>}{\end{column}}

%% Add footer with logo
\setbeamertemplate{footline}{%
  \begin{beamercolorbox}[wd=\paperwidth,ht=2.5ex,dp=1.125ex,%
    leftskip=.3cm,rightskip=.3cm plus1fil]{author in head/foot}
    \includegraphics[height=5ex]{SIAP_logo_Big.png}\hfill
    \insertshortauthor\hfill\insertshorttitle\hfill  \textcolor{siap}{\textit{\insertframenumber}}
  \end{beamercolorbox}%
}


% Path for the graphs
\graphicspath{{Graphics/}
{../../../../Visualisation/Presentations/Graphics/}
{../../Visualisation/Presentations/Graphics/}
{c:/Chris/Visualisation/Presentations/Graphics/Logos/}
{c:/Gitmain/MLCourse/UNML/Module2/M2-1-SimpleClassification_files/figure-html/}
 }
% Path for the graphs
\graphicspath{{Graphics/}
{../../../../Visualisation/Presentations/Graphics/}
{../../Visualisation/Presentations/Graphics/}
{c:/Gitmain/MLCourse/UNML/Module0/M0_files/figure-html/}
{c:/Chris/UN-ESCAP/MyCourses2022/MLOS2022/Slides/Graphics/}
{c:/Chris/UN-ESCAP/MyCourses2023/BigDataKostat/Slides/Graphics/}
 }

%remove navigation symbols
\setbeamertemplate{navigation symbols}{}


% Natbib for clean bibliography
\usepackage[comma,authoryear]{natbib}

%%%% Use it or not %%%%

\title{\textcolor{siap}{Machine Learning for Official Statistics in Asia-Pacific countries \\ \vspace{0.5cm} }}

\subtitle{\textcolor{brique}{\Large{Classification}}}
\author{\textcolor{siap}{Christophe Bontemps}}
\institute{ \includegraphics[height=9ex]{SIAP_logo_Big.png}}
\date{}

\begin{document}

\begin{frame}
  \titlepage
\end{frame}


\section{Introduction}

\begin{frame}

\frametitle{\textcolor{brique}{[ Classification ]}}
What is a classification problem?
\pause
\begin{itemize}[<+->]
  \item The goal of classification is to understand why an observation belongs to a certain category
  \item The interest is on $y$ that takes discrete values:  0/1, high school/prilmary school/no education; urban/rural
  \item There may be variables explaining why we observe that $y$ belongs to a particular category
  \item A classifier is a tool that  provides a classification for $y$
\end{itemize}
\end{frame}

\begin{frame} % Cover slide
\frametitle{\textcolor{brique}{[ Supervised \textit{vs} unsupervised Classification ]}}
\pause
\begin{itemize}[<+->]
  \item In \textbf{supervised }classification, we observe the class for  $y$
  \item[] One may learn from that information and estimate the impact of other variables on that classification (\textit{e.g}. logit regression)
   \item In \textbf{unsupervised }classification, we observe a set of variables for each observation
  \item[]  The goal is to classify observations from those variables (clustering) without having any information of what a category means.
  \item[]
  \item We'll focus on \textbf{supervised }classification
\end{itemize}
\end{frame}

\begin{frame} % Cover slide
\frametitle{\textcolor{brique}{[ Classification:  An example ]}}
\pause
\begin{itemize}[<+->]
  \item You observe households that are either in \textit{Urban} or \textit{Rural} areas (colors) and one variable (feature): \textit{Education}.
  \item[] \begin{center}\includegraphics[width = 0.5\textwidth]{univariatefig-1.png} \end{center}
  \item A classifier determines the value of\textit{ Education} that separate "Rural" from  "Urban", typically
 with a threshold rule : "if $x \geq x_t$  then $y$ is categorized as Urban"
\end{itemize}
\end{frame}

\begin{frame} % Cover slide
\frametitle{\textcolor{brique}{[ Classification:  A 2-D example ]}}
\pause
\begin{itemize}[<+->]
  \item You observe households that are either in \textit{Urban} or \textit{Rural} areas (colors) and \textbf{two} variables (features): \textit{Education} and \textit{Income}
  \item[] \begin{center}\includegraphics[width = 0.5\textwidth]{bivariatefig-1.png} \end{center}
\end{itemize}
\end{frame}

\begin{frame} % Cover slide
\frametitle{\textcolor{brique}{[ Classification:  A 2-D example ]}}
\pause
\begin{itemize}[<+->]
  \item[] \begin{center}\includegraphics[width = 0.3\textwidth]{bivariatefig-1.png} \end{center}
   \item A classifier will determine a\textbf{ boundary} using both \textit{Education}  and \textit{Income} to separate "Rural" from "Urban"
  \item The rule can be based on a linear relationship between \textit{Education}  and \textit{Income} or can be non linear.
\end{itemize}
\end{frame}

\begin{frame} % Cover slide
\frametitle{\textcolor{brique}{[ Classification:  A 2-D example ]}}
\pause
\begin{itemize}[<+->]
  \item The logit is an example of linear classifier
  \item[] \begin{center}\includegraphics[width = 0.5\textwidth]{Logit_Frontier-1.png} \end{center}
  \item The rule that separated the two classes is $ x'\beta \geq T_0$ with $T_0$ a known threshold
  \item[]\textit{e.g.} $\beta_1 Education + \beta_2 Income \geq T_0$

\end{itemize}
\end{frame}

\begin{frame} % Cover slide
\frametitle{\textcolor{brique}{[ Classification:  A 2-D example ]}}
\pause
\begin{itemize}[<+->]
  \item The quadratic logit is an example of non-linear classifier
  \item[] \begin{center}\includegraphics[width = 0.5\textwidth]{QLogit_Frontier-1.png} \end{center}
  \item The rule that separated the two classes is non linear in the variables \textit{Education}  and \textit{Income}
\end{itemize}
\end{frame}

\begin{frame} % Cover slide
\frametitle{\textcolor{brique}{[ Classification:  A 2-D example ]}}
\pause
\begin{itemize}[<+->]
  \item Other examples can be very non linear
  \item[] \begin{center}\includegraphics[width = 0.5\textwidth]{PlotBayesK-1.png} \end{center}
  \item It is hard to understand how the two classes  are built using \textit{Education}  and \textit{Income}
\end{itemize}
\end{frame}

\begin{frame} % Cover slide
\frametitle{\textcolor{brique}{[ Classification:  A 2-D example ]}}
\pause
\begin{itemize}[<+->]
  \item Other examples can be very non linear
  \item[] \begin{center}\includegraphics[width = 0.5\textwidth]{PlotQDA-1.png} \end{center}
  \item The boundary is very complex but uses \textit{Education}  and \textit{Income} features.
\end{itemize}
\end{frame}

\section{Measures of Fit}

\begin{frame} % Cover slide
\frametitle{\textcolor{brique}{[ How to select the right model?  ]}}
\pause
\begin{itemize}[<+->]
  \item What is the goal?
  \item[] Have the ``best`'' classification
  \item[$\hookrightarrow$] Need for a criterion to determine what is a good classifier
  \item Measures of fit in classification are different and specific
\end{itemize}
\end{frame}

\begin{frame} % Cover slide
\frametitle{\textcolor{brique}{[ Measures of fit in classification ]}}
There are several popular measures of fit, differing in their spirit and their goal
\pause
\begin{itemize}[<+->]
  \item Accuracy
  \item Confusion matrix
  \item Specificty \& sensitivity
  \item Kappa
  \item[]$\cdots$
  \item[] All answer to different questions and solve different problems
\end{itemize}
\end{frame}

\begin{frame} % Cover slide
\frametitle{\textcolor{brique}{[ Accuracy and Confusion Matrix]}}
\begin{itemize}[<+->]
  \item[] Accuracy corresponds to the probability of being "accurate"
        $$
        \Pr \left[ y_0  = \widehat{f}(x_0) \right]
        $$
    \item where $\widehat{f}(\cdot)$ is the classifier.
    \item[$\hookrightarrow$] We want the maximum possible accuracy.
    \item Equivalently, we may want to minimize the \textit{error rate} or \textit{misclassification rate}
    $$
    \Pr \left[ y_0  \neq \widehat{f}(x_0) \right]
    $$
\end{itemize}
\end{frame}

\begin{frame} % Cover slide
\frametitle{\textcolor{brique}{[ Confusion Matrix \& Accuracy ]}}

A classifier predicts in which class each observation should be:
\pause
\begin{itemize}[<+->]
  \item[]
    \begin{table}[]
    \begin{tabular}{l|c|c|}
                               & \multicolumn{2}{c|}{Observed (True)}                                                                                          \\ \hline
    \multirow{2}{*}{Predicted} & \begin{tabular}[c]{@{}l@{}}TP\\ (True Positive)\end{tabular}  & \begin{tabular}[c]{@{}l@{}}FP\\ (False Positive)\end{tabular} \\ \cline{2-3}
                               & \begin{tabular}[c]{@{}l@{}}FN\\ (False Positive)\end{tabular} & \begin{tabular}[c]{@{}l@{}}TN\\ (True Negative)\end{tabular}  \\ \hline
    \end{tabular}
    \caption{Confusion Matrix}
    \end{table}

  \item Accuracy is then the ratio:
   $$ Acuracy = \frac{TP + TN}{ TP+TN+FP+FN} $$
   $$
 \; \; \;  =  \frac{ True Positives +  True Negatives}{N}
 $$

  \item  It is the proportion of accurate predictions
\end{itemize}
\end{frame}

\begin{frame} % Cover slide
\frametitle{\textcolor{brique}{[Confusion Matrix  \& Accuracy]}}
In practice, with a classifier we have:
\begin{itemize}[<+->]
  \item[]
    \begin{table}[]
    \begin{tabular}{l r|c|c|}

                              & & \multicolumn{2}{c|}{Observed (True)}                                                                                          \\
                              & &    Urban  &  Rural  \\  \hline
    \multirow{2}{*}{Predicted}& Urban & \shortstack[c]{$87$ \\ \tiny{(TP)}}& \shortstack[c]{28 \\ \tiny{(FP)}} \\ \cline{2-4}
                              & Rural &  \shortstack[c]{24 \\ \tiny{(FN)}}& \shortstack[c]{69 \\ \tiny{(TN)}} \\ \hline
    \end{tabular}
    \caption{Confusion Matrix}
    \end{table}
  \item Remember that here \textit{Urban} is the "positive" class
  \item Accuracy is then the ratio:
   $$ Acuracy = \frac{87 + 69}{ 87 + 69 + 28 + 24} $$
   $$
 \; \; \;  =  \frac{156}{208} = 0.75
 $$
  \item  We have an accurate prediction in 75\% of the cases.
\end{itemize}
\end{frame}

\begin{frame} % Cover slide
\frametitle{\textcolor{brique}{[Confusion Matrix  \& Accuracy]}}
Accuracy is not the panacea and may be misleading
\begin{itemize}[<+->]
    \item[\textbf{Pb 1}:] Imagine a data set where 95\% of the observations are in class 1 and 5\% in the remaining class
    \item[$\hookrightarrow$] A  classifier that predicts always class 1 will have an accuracy of 95\%!
    \item Accuracy is not very useful with an imbalanced data set and one may prefer another indicator: \textit{kappa}
    \item[\textbf{Pb 2}:] One may be more interested in correctly predicting a particular outcome
    \item[$\hookrightarrow$] This is often the case
    \item One may need other measures, such as \textit{Specificity} or \textit{Sensitivity}
\end{itemize}
\end{frame}

\begin{frame} % Cover slide
\frametitle{\textcolor{brique}{[Sensitivity or \textit{True Positive Rate}]}}
\begin{itemize}[<+->]
  \item[]
    \begin{table}[]
    \begin{tabular}{l r|c|c|}

                              & & \multicolumn{2}{c|}{Observed (True)}                                                                                          \\
                              & &    Urban  &  Rural  \\  \hline
    \multirow{2}{*}{Predicted}& Urban & \shortstack[c]{$87$ \\ \tiny{(TP)}}& \shortstack[c]{28 \\ \tiny{(FP)}} \\ \cline{2-4}
                              & Rural &  \shortstack[c]{24 \\ \tiny{(FN)}}& \shortstack[c]{69 \\ \tiny{(TN)}} \\ \hline
    \end{tabular}
    \caption{Confusion Matrix}
    \end{table}
  \item \textit{Sensitivity} focuses on  "positives" (here \textit{Urban}), \textit{i.e.} on predicted positives \textit{vs} the observed positives
   $$ Sensitivity = \frac{TP}{TP + FN} $$
   $$
          \hspace{2.5cm}=  \frac{87}{87+24} = 0.78
 $$
  \item On \textit{Urban}, we correctly predict  in 78\% of the cases
\end{itemize}
\end{frame}

\begin{frame} % Cover slide
\frametitle{\textcolor{brique}{[Specificity or \textit{True Negative Rate}]}}
\begin{itemize}[<+->]
  \item[]
    \begin{table}[]
    \begin{tabular}{l r|c|c|}

                              & & \multicolumn{2}{c|}{Observed (True)}                                                                                          \\
                              & &    Urban  &  Rural  \\  \hline
    \multirow{2}{*}{Predicted}& Urban & \shortstack[c]{$87$ \\ \tiny{(TP)}}& \shortstack[c]{28 \\ \tiny{(FP)}} \\ \cline{2-4}
                              & Rural &  \shortstack[c]{24 \\ \tiny{(FN)}}& \shortstack[c]{69 \\ \tiny{(TN)}} \\ \hline
    \end{tabular}
    \caption{Confusion Matrix}
    \end{table}
  \item \textit{Specificity} focuses on  negatives (\textit{Rural}), \textit{i.e.} on predicted negatives \textit{vs} the observed negatives
   $$ Specificity = \frac{TN}{TN + FP} $$
   $$
 \; \; \;  =  \frac{69}{69+28} = 0.71
 $$
  \item On \textit{Rural}, we predict correctly in \textbf{only} 71\% of the cases
\end{itemize}
\end{frame}

\begin{frame} % Cover slide
\frametitle{\textcolor{brique}{[Confusion Matrix  \& Kappa ($\kappa$) ]}}
\emph{Kappa} ($\kappa$) is defined to measure the accuracy with imbalanced classes
\begin{itemize}[<+->]
  \item[] Its formal definition is given by
  $$ \kappa = \frac{P_o - P_e}{1 - P_e}  $$
  \item[]$P_o$ is the current classifier accuracy which is compared here with the accuracy of an uniformed classifier $P_e$
  \item[] $P_e$ is the accuracy of a classifier that would operate purely by chance, using no information.
  \item $P_o$ is simple (here = 0.75) while  $P_e$  is more complex to compute.
  \item  The larger $\kappa $ is, the better the model for a given distribution of classes in a data set
  \end{itemize}
\end{frame}


\section{Logit}

\begin{frame} % Cover slide
\frametitle{\textcolor{brique}{[Logit as You know it ]}}
\pause
 \begin{itemize}[<+->]
  \item[] When dealing with a discrete outcome $y$ we cannot use a $direct$ linear relationship between $y$ and the explanatory variables $x$ ( \textit{i.e. Education, Income})
  \item Logit aims at estimating the probabilities  $\pi = Probablity[y= 1]$, which are continuous ($\in [0,1]$)
  \item The definition of the logit model is:
  $$
  \pi = Pr \left( y = 1\right) = F( x'\beta) = \frac{1}{1+\exp(-x'\beta)}
  $$
  \item This can be transformed into:
  $$
  \pi = \frac{exp(-x'\beta)}{1+ exp(-x'\beta)}
  $$
\end{itemize}
\end{frame}

\begin{frame} % Cover slide
\frametitle{\textcolor{brique}{[Logit as You \textbf{Don't} know it ]}}
\pause
 \begin{itemize}[<+->]
  \item[] From this equation
  $$
  \pi = \frac{exp(-x'\beta)}{1+ exp(-x'\beta)}
  $$
  \item[] one gets  the \textbf{linear} nature of the logit:
   $$
  log(\frac{\pi}{1 - \pi}) = x'\beta
  $$
  \item[]where $\frac{\pi}{1 - \pi}$ is the odd ratio $\in [0,\infty]$  with values indicating high or low probability that $y=1$
  \item[$\hookrightarrow$] "\emph{The logit models log of odd ratios as linear in x}"
\end{itemize}
\end{frame}

\begin{frame} % Cover slide
\frametitle{\textcolor{brique}{[Logit as a classifier ]}}
\pause
 \begin{itemize}[<+->]
  \item Once estimated, $\widehat \pi_i$ provide a simple rule for classification
  \item[] $$
   \widehat \pi_i > t_0  \Leftrightarrow \widehat y_i = 1
  $$
    \item[] Where $t_0$ is a threshold provability, by default $1/2$.
  \item If  $t_0 = 1/2$ (default), then the rule is equivalent to:
   $$
   x'_i \widehat \beta > 0  \Leftrightarrow \widehat y_i = 1
  $$
  \item If  $t_0 \neq 1/2$:
   $$
   x'_i \widehat \beta > T_0  \Leftrightarrow \widehat y_i = 1
  $$
  \item[$\hookrightarrow$] The logit classifier depends on the linear combination of the $x$'s
\end{itemize}
\end{frame}

\begin{frame} % Cover slide
\frametitle{\textcolor{brique}{[Importance of the threshold ]}}
\pause
 \begin{itemize}[<+->]
  \item The rule $ x'\beta \geq T_0$ defines the partition of the space
  \item[] \begin{center}\includegraphics[width = 0.5\textwidth]{Logit_Frontier-1.png} \end{center}
  \item This partition is sensitive to the choice of the threshold $T_0$ (and the $t_0$)
\end{itemize}
\end{frame}

\begin{frame} % Cover slide
\frametitle{\textcolor{brique}{[Importance of the threshold ]}}
\pause
 \begin{itemize}[<+->]
  \item Changing $t_0$ will  change the predictions \&  the classification
  \item[] A higher $t_0$ will allocate less observations to the $y=1$ category (Urban)
  \item[] A lower $t_0$ will allocate more observations to the $y=1$ category
  \item The choice of $t_o$ should be done according to the data and observed classes repartition
  \item \textit{Specificit}y and \textit{Sensitivity} are affected by $t_0$
\end{itemize}
\end{frame}

\begin{frame} % Cover slide
\frametitle{\textcolor{brique}{[The ROC Curve ]}}
\pause
 \begin{itemize}[<+->]
  \item We want the \textit{Specificit}y and \textit{Sensitivity} to be both \textbf{maximized} (ideally both would be 1)
  \item The ROC curve help visualize the best choice
  \item[] \textit{Be careful of the axes }
  \item
\end{itemize}
\end{frame}

\begin{frame} % Cover slide
\frametitle{\textcolor{brique}{[The ROC Curve]}}
\pause
 \begin{itemize}[<+->]
  \item[] The ROC represents values of 1- Specificity =  FPR  \textit{vs} Sensitivity =  TPR for many values of the threshold $t_0$
  \item[] \begin{center}\includegraphics[width = 0.6\textwidth]{ROCfig-1.png} \end{center}
  \item \textit{(sometimes  x is sensitivity  with inverted x-axis)}
\end{itemize}
\end{frame}

\begin{frame} % Cover slide
\frametitle{\textcolor{brique}{[The ROC Curve: How to read?]}}
\pause
 \begin{itemize}[<+->]
  \item[] Changing $t_0$ changes the classification of observations
  \item  Optimally, the curve should touch top-left corner
  \item[] \begin{center}\includegraphics[width = 0.6\textwidth]{ROCfig-1.png} \end{center}
  \item If $t_0  \nearrow $,  more cases  classified as \textit{Negatives}, less \textit{Positives}
  \item If $t_0  \nearrow $,   specificity  $\nearrow $ and sensitivity $\searrow$
\end{itemize}
\end{frame}

\begin{frame} % Cover slide
\frametitle{\textcolor{brique}{[AUC as a measure of fit]}}
\end{frame}

\section{Best classifier}

\begin{frame} % Cover slide
\frametitle{\textcolor{brique}{[ Train \& Validation set ]}}
\end{frame}

\begin{frame} % Cover slide
\frametitle{\textcolor{brique}{[ CV on measures of fit ]}}
\end{frame}

\begin{frame} % Cover slide
\frametitle{\textcolor{brique}{[ Model comparison]}}
\end{frame}

\section{wrap-up}

\begin{frame} % Cover slide
\frametitle{\textcolor{brique}{[Takeways]}}
\begin{itemize}[<+->]
  \item In classification, the \textbf{Confusion matrix} is important
 \item Many adjustment measures:  \textbf{accuracy}, \textbf{sensitivity} and \textbf{specificity}.
    \begin{itemize}[<+->]
      \item Sensitivity is accuracy restricted to the positives.
      \item Specificity is accuracy restricted to the negatives.
   \end{itemize}
 \item When  outcome is \textit{imbalanced}, one may use \textbf{kappa} has a better measure for accuracy.
  \item[] Which measure you should consider depends on the context and your goal.
 \item \textbf{Logit} is a benchmark parametric model for classification
 \item[] One may use the \textbf{ROC} to change the threshold parameter
 %\item[] \textit{(don't forget to scale explanatory variables) }.
\end{itemize}
\end{frame}

\begin{frame} % Cover slide
\frametitle{\textcolor{brique}{[Takeways]}}
\begin{itemize}[<+->]
  \item Use Training-Validation set to \textbf{select} parameters within a model
  \item Use Training-Validation set to \textbf{compare} models on the same criteria
  \item Several criteria / measures of fit / cost functions are available
  \item Time is the limit...
\end{itemize}
\end{frame}

%%%%%%%%%%%%%%%%%%%%%%%%%%%%%%%
\end{document}


%%%%%%%%%%%%%%%%%%%%%%%%%%%%%%%
\begin{frame} % Cover slide
\frametitle{ }
\pause
 \begin{itemize}[<+->]
  \item[]
  \item
\end{itemize}
\end{frame}

%%%%%%%%%%%%%%% Last Slide %%%%%%%%%%%%%%%%

\begin{frame}[allowframebreaks]%in case more than 1 slide needed
\frametitle{References}
    {\footnotesize
    %\bibliographystyle{authordate1}
    \bibliographystyle{apalike}
    \bibliography{../../../Visualisation/Visu}
    }
\end{frame}
\end{document}

%\bibliographystyle{authordate1}
%\bibliography{c:/Chris/Visualisation/Visu}
%\end{frame}
