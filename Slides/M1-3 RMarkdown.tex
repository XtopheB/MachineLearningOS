%%%%%%%%%%%%%%%%%%%%%%%%%%%%%%%%%%%%%%%%%%%%%%%%%%%%%%%%%%%
%\documentclass[xcolor=x11names,compress]{beamer}
\documentclass[xcolor=x11names,compress]{beamer}
%% General document
\usepackage{graphicx, subfig}
%% Beamer Layout
\useoutertheme[subsection=false,shadow]{miniframes}
\useinnertheme{default}
\usefonttheme{serif}
\usepackage{palatino}

%%%%%%% Mes Packages %%%%%%%%%%%%%%%%
%\usepackage[french]{babel}
\usepackage[T1]{fontenc}
\usepackage{color}
\usepackage{xcolor}
\usepackage{dsfont} % Pour indicatrice
\usepackage{url}
\usepackage{multirow}
\usepackage[normalem]{ulem}   % For strike out text

% Natbib for clean bibliography
\usepackage[comma,authoryear]{natbib}

%remove the icon
\setbeamertemplate{bibliography item}{}

%remove line breaks
\setbeamertemplate{bibliography entry title}{}
\setbeamertemplate{bibliography entry location}{}
\setbeamertemplate{bibliography entry note}{}

%% ------ MEs couleurs --------
\definecolor{vert}{rgb}{0.1,0.7,0.2}
\definecolor{brique}{rgb}{0.7,0.16,0.16}
\definecolor{gris}{rgb}{0.7, 0.75, 0.71}
\definecolor{twitterblue}{rgb}{0, 0.42, 0.58}
\definecolor{airforceblue}{rgb}{0.36, 0.54, 0.66}
\definecolor{siap}{RGB}{3,133, 200}


%%%%%%%%%%%%%%%%% BEAMER PACKAGE %%%%%%%

\setbeamercolor{itemize item}{fg=siap}
%\setbeamercolor{itemize subitem}{fg=blue}
%\setbeamercolor{itemize subsubitem}{fg=cyan}

\setbeamerfont{title like}{shape=\scshape}
\setbeamerfont{frametitle}{shape=\scshape}

\setbeamercolor*{lower separation line head}{bg=DeepSkyBlue4}
\setbeamercolor*{normal text}{fg=black,bg=white}
\setbeamercolor*{alerted text}{fg=siap}
\setbeamercolor*{example text}{fg=black}
\setbeamercolor*{structure}{fg=black}
\setbeamercolor*{palette tertiary}{fg=black,bg=black!10}
\setbeamercolor*{palette quaternary}{fg=black,bg=black!10}

% Set the header color to SIAP's color
\setbeamercolor*{frametitle}{fg=siap}

%remove navigation symbols
\setbeamertemplate{navigation symbols}{}

\renewcommand{\(}{\begin{columns}}
\renewcommand{\)}{\end{columns}}
\newcommand{\<}[1]{\begin{column}{#1}}
\renewcommand{\>}{\end{column}}

%% Add footer with logo
\setbeamertemplate{footline}{%
  \begin{beamercolorbox}[wd=\paperwidth,ht=2.5ex,dp=1.125ex,%
    leftskip=.3cm,rightskip=.3cm plus1fil]{author in head/foot}
    \includegraphics[height=5ex]{SIAP_logo_Big.png}\hfill
    \insertshortauthor\hfill\insertshorttitle\hfill  \textcolor{siap}{\textit{\insertframenumber}}
  \end{beamercolorbox}%
}


% Path for the graphs
\graphicspath{{Graphics/}
{../../../../Visualisation/Presentations/Graphics/Logos}
{../../Visualisation/Presentations/Graphics/}
{c:/Chris/UN-ESCAP/MyCourses2024/BigData/Slides/Graphics/}
{c:/Gitmain/MLCourse/UNML/Module0/M0_files/figure-html/}
{c:/Chris/UN-ESCAP/MyCourses2022/MLOS2022/Slides/Graphics/}
{c:/Chris/UN-ESCAP/MyCourses2023/BigDataKostat/Slides/Graphics/}
{c:/Chris/UN-ESCAP/MyCourses2024/WebScrapping/Graphics/}
 }

\title{\textcolor{siap}{Big Data: Innovative Methods and Applications for Achieving SDGs \\ \vspace{0.5cm} }}

\subtitle{\textcolor{brique}{\Large{A crash course in R Markdown}}}
\author{Christophe Bontemps}
\institute{ \includegraphics[height=10ex]{SIAP_logo_Big.png}}
\date{}

\begin{document}

% Slide 1: Title Slide
\begin{frame}
    \titlepage
\end{frame}

\section{What is R Markdown}

% Slide 2: What is R Markdown?
\begin{frame}
    \frametitle{What is R Markdown?}
    \begin{columns}
        \column{0.7\textwidth}
            \begin{itemize}[<+->]
                \item R Markdown is a file format used natively in RStudio.
                \item It integrates text, code, and results into one document.
                \item Supports reproducibility by combining code and analysis.
            \end{itemize}
        \column{0.3\textwidth}
            \includegraphics[width=0.5\textwidth]{Rmarkdown-logo.png} % Example R Markdown image
    \end{columns}
\end{frame}

\section{Key components}
% Slide 3: Key Components of R Markdown
\begin{frame}
    \frametitle{Key Components of R Markdown}
    \begin{columns}
         \begin{column}{0.35\textwidth}
            \begin{itemize}[<+->]
                \item YAML Header
                \item Code Chunks
                \item Text/Markdown
                \item Output Formats
            \end{itemize}
        \end{column}
        \begin{column}{0.65\textwidth}
            \begin{itemize}
            \item[]
                 \only<1>{\includegraphics[width=\textwidth]{MyYAML.png}}
                \only<2-3>{\includegraphics[width=\textwidth]{MyCodeChunks.png}}
                \only<4>{\includegraphics[width=0.5\textwidth]{format-dropdown.png}}
            \end{itemize}
         \end{column}
    \end{columns}
\end{frame}


% Slide 4: YAML Header
\begin{frame}
    \frametitle{YAML Header}
    \begin{itemize}[<+->]
        \item Appears at the top of the R Markdown file.
        \item Defines metadata like:\\
         title, author, date, and output format.
        \item[] Example: \\
        \includegraphics[width=\textwidth]{MyYAML.png} % YAML image
   \end{itemize}
\end{frame}
% Slide 5: Code Chunks

\begin{frame}
    \frametitle{Code Chunks}
    \begin{itemize}[<+->]
        \item Code chunks allow you to embed R code inside the document.
        \item Chunks are enclosed by \textbf{3} backticks.
        \item Example:
      \item[]
       \includegraphics[width=\textwidth]{ChunkExamples.png}

    \end{itemize}
\end{frame}


% Slide 6: Chunk Options
\begin{frame}
    \frametitle{Chunk Options}
    \begin{columns}
        \column{0.5\textwidth}
            \begin{itemize}[<+->]
                \item Control how code chunks behave.
                \item Common options:
                \item[-] echo = TRUE/FALSE
                \item[-] eval = TRUE/FALSE
                \item[-] include = TRUE/FALSE

            \end{itemize}
        \column{0.5\textwidth}
            \includegraphics[width=\textwidth]{ChunkOptions.png}
    \end{columns}
\end{frame}

\section{Syntax}
%%% https://rmarkdown.rstudio.com/lesson-1.html
% Slide 7: Markdown Syntax
\begin{frame}
    \frametitle{Markdown Syntax}
    \begin{columns}
        \column{0.5\textwidth}
            \begin{itemize}[<+->]
                \item Use simple markdown syntax to format text.
                \item Examples:
                \begin{itemize}
                    \item *Italics*: *text*
                    \item **Bold**: **text**
                    \item `Code`: `code`
                \end{itemize}
            \end{itemize}
        \column{0.5\textwidth}
            \includegraphics[width=\textwidth]{markdowlanguage.png}
    \end{columns}
\end{frame}

\section{Conclusion}
\begin{frame}
    \frametitle{Conclusion and Resources}
    R Markdown integrates text and code for reproducible research.
    \begin{columns}
        \column{0.5\textwidth}
            \begin{itemize}[<+->]
                \item For more resources:
                \item[-]\href{https://rmarkdown.rstudio.com}{R Markdown website}
                \item[-] \href{https://r4ds.had.co.nz/}{R for Data Science}
            \end{itemize}
        \column{0.5\textwidth}
            \href{https://rmarkdown.rstudio.com/lesson-2.html}{\includegraphics[width=\textwidth]{RmarkdownManual.png}}
    \end{columns}
\end{frame}

% the \setbeamercolor and the frame to limit the scope
{\setbeamercolor{background canvas}{bg=siap}
 \setbeamercolor{item}{fg=white}
    \setbeamercolor{normal text}{fg=white}
    \usebeamercolor[fg]{normal text}

\begin{frame}
\hspace{4cm}
\begin{center}
\Huge{\href{https://github.com/XtopheB/BigData/blob/main/ReportExercise.Rmd}{Your Turn}}
\end{center}
\end{frame}

} %end blue Background



\end{document}






